\subsection{Interactions and Linkage}
To better facilitate the interactive exploration of RNNs, our system supports cross-view interactions. In this section, we summarize the interactions and linkage among all four views.

\textbf{Cross-view highlight.} 
In summary, there are three key visualization components appearing across different views: \textbf{feature}, \textbf{feature cluster} and \textbf{case}. These components are visualized in different forms in different views to support various analysis requirements. For example, a feature is visualized as a glyph (Fig.\ref{fig:cluster_design}E) in the cluster view,  visualized as a series of boxplots (Fig.\ref{fig:teaser}C) in the Feature Importance View and shown as both a line chart (Fig.\ref{fig:individual_view}A$_1$) and a sequence (Fig.\ref{fig:individual_view}A$_3$) in the individual view. If one feature is selected by mouse hover, the corresponding visualization element in other views will be highlighted with a border stroke.

\textbf{Linkage between individual view and feature importance view.} 
When multiple individual cases are selected, in addition to visualizing these cases in individual views, the feature importance by time will be visualized as a line chart in the corresponding feature importance views as Fig.\ref{fig:teaser}C shows. When the user chooses an individual case by mouse hover, the corresponding line-chart will be highlighted by border stroke width. If the user selects multiple individual cases using check boxes, the corresponding line charts will be highlighted in red color.


