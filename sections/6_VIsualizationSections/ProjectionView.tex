%!TEX root = ../article.tex
\subsection{Projection View}
To help users obtain an overview of case clusters and outliers (\textbf{T5}), we design the Projection View (Fig.~\ref{fig:teaser}E) which supports various interactions such as zooming and brushing to allow users to select a subset of data for further examination.

In the Projection View, each circle represents a individual case. There are many multi-dimensional reduction methods such as MDS and PCA; we select t-SNE as it can strongly repel dissimilar points and show clusters clearly.
For each case, we collect the feature cluster importance over all time steps (discussed in Sec.\ref{section:feature_importance}) as the input vectors of t-SNE.
% \QM{Given an individual case, we flatten the cluster importance at all timestamps (discussed in Sec.\ref{section:feature_importance}) to 1-dimensional as the t-SNE embedding vector.}
% \yh{?}
Thus, the positions of the circles reflect the similarity of their cluster importance.
\QM{We use a sequential color to encode the model's output of each case shown as the legend in Fig.~\ref{fig:teaser}E.}
%\yh{$<$to discuss: sequential color scheme description?$>$}
% Add to increase the space
% When multiple data subsets are selected, we use a categorical color scheme to fill the circles so that the data sequences from the same data subset will have the same color in both components.
% Users can brush to select data sequences for detailed examination in the Sequence View.

% One major consideration in developing the Projection View is the scalability problem.
% When the number of circles is large, the circles may overlap with each other, and the Projection View can be cluttered.
% We adopt a node overlap removal algorithm for the similarity projection and support panning and zooming to allow users to explore a data subset.

Furthermore, to improve the flexibility of the case selection, we add a two-scale timeline (Fig.~\ref{fig:teaser}E~top) to show the target feature trend, enabling user filtering of the cases by time, and a feature selection component (Fig.~\ref{fig:teaser}E~left) to filter the cases by feature value.  

