\subsection{Feature Importance View}
The feature importance view allows users to explore the feature contribution to model output (\textbf{T2}).
As discussed in Sec.\ref{section:feature_importance}, with an input case, we are able to measure the importance of a single feature as a sequence of importance scores which correspond to the importance at all timestamps (Fig.~\ref{fig:teaser}C). 

Since the importance score only provides a local description for the feature importance, an effective visualization is needed to show an overview of each feature's importance. We choose boxplot for this task since it can present the statistics overview. To show the temporal trend of a feature, we group the importance score of all test cases by the timestamps and make statistics group by group. For the test sequence with a length $T$, the feature importance charts which contain $T$ boxplots shows the trend of feature importance (Fig.~\ref{fig:teaser}C).

The horizontal axis indicates the timestamps and the vertical axis indicates the feature importance score. The top line, upper edge, middle line, bottom edge and bottom line of the boxplot indicate the maximum,  $75^{th}$ percentile, mean, $25^{th}$ percentile and minimum of the importance scores.  Since sometimes the maximum will much larger than the $75^{th}$  percentile value, which makes the box vary flat and difficult for users to explore the temporal pattern, we limit the maximum score $Ms$ shown in the view. If a boxplot has scores larger than $Ms$, a diamond symbol appears on the top of the boxplot. The opacity of the diamond indicates the magnitude of the absolute difference between the largest score and $Ms$.    

We also define the overall importance score for a single feature as the sum of the mean score at all timestamps. By default, the boxplot charts will be ranked according to the overall feature importance score. 
Due to the large number of features, only the top 10 charts are visualized. Users may observe other features by using the scroll bar or filtering the features from the projection view (Fig.~\ref{fig:teaser}E). 

