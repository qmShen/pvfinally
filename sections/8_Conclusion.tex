\section{Discussion and Conclusion}

In this paper, we presented MultiRNNExplorer, a visual analytic system for the understanding of RNN models for high-dimensional time-series forecasting, and we use air pollutant forecasting as the target application. 
To describe the global level of the model mechanism, we propose a technique to estimate the hidden units' response to an individual feature by measuring how different feature selections affect the hidden units' output distribution. At the individual level, we further use the gradient-based method to measure the local feature importance case by case. Based on these techniques, we design a visual analytic system which enables the users to explore and reason about the model behavior from different perspectives. The evaluation includes three case studies which demonstrate the effectiveness of the proposed system for comprehensive analysis of RNNs.There are still some issues about our work which need to be discussed:

\textbf{Visual scalability.} Several views suffer scalability issues with the increasing number of cases. In Projection View, thousands of individual cases need to be visualized and cause serious visual clutter due to the overlap of circles.  Even though we provide interactions such as zoom and pan which enable the users to iteratively explore the cases, it is still difficult for the users to clearly observe the overview of the case similarity.  In our case, more than 8000 points are visualized. If data size keeps increasing, we need to sample or group the cases before generating the projection. We may also apply other advanced projection techniques\cite{van2017visual, pezzotti2016hierarchical} for Projection View. 
The individual view also has such a problem: it is easy for users to brush hundreds of individual cases from Projection View and generate tens of clusters. Due to the limited screen size, our current design allows 9 groups of individual cases to be shown at the same time and uses the scroll bar to enable the observation of more groups.
Throughout the system, the feature type is encoded by categorical colors. Because humans suffer perception issues for more than 10 categorical colors, we manually choose similar colors for highly correlated features (for example, $PM_{2.5}$ and $PM_{10}$ are encoded by light green and dark green). For more features, we may need only a few colors for the features of interest and encode other features by the same color (for example, gray). 

\textbf{Generalization.}
Although our target application is air pollutant forecasting, our proposed method can be easily extended to other high-dimensional time-series forecasts with very few changes. The major design which needs to be revised to adapted the application is the feature glyph (Fig.\ref{fig:cluster_design}). The current design is able to encode three attributes about the spatial data including direction, type and distance. Any other cell-like glyph design is also applicable in our method. 


There is some promising future work for MultiRNNExplorer. First, improve the efficiency of the response estimation. With 8375 test cases, the current method needs hours of calculation which has to be done offline. 
In the future work, we will improve this method by introducing rational sampling techniques or develop alternative estimation techniques to enable our system to support realtime data processing. Another improvement of our system is to enhance the individual comparison. In the current design, the individual comparison has to be conducted by comparing the individuals side by side. A specific comparison design\cite{von2015mobilitygraphs} will be helpful to ease the burden of users. We also consider to use our system on other high-dimensional forecasting applications including classification tasks such as fraud detection\cite{jurgovsky2018sequence}. We are also exploring the possibility of applying some methods proposed in this work to audio modeling tasks. 

