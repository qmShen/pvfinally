%!TEX root = ../article.tex
% \section{Introduction}
\label{section:introduction}
Recurrent neural networks (RNNs) have been widely applied in various natural language processing (NLP) tasks such as machine translation and sentiment analysis. 
Compared with other network architectures like fully connected networks, RNNs can effectively model sequential data.
Benefiting from this, RNNs have been extended to other domains of sequential data beyond NLP, such as to forecast weather\cite{xingjian2015convolutional, cao2012forecasting, shi2017deep}, air pollutant\cite{oprea2016neural, zhou2017prediction, li2017long}, and even stock market\cite{kim2018forecasting,nelson2017stock,akita2016deep}.
Compared to NLP tasks which take latent embeddings as inputs, the input features in these applications are usually multi-dimensional time series where each dimension has its own semantic meaning.
In air pollutant forecast, as a focus of this work, the inputs are hourly recorded series of multi-dimensional pollutants (\emph{e.g.}, $SO_{2}$) and influences (\emph{e.g.}, \textit{wind speed}).

Despite their competitive performance, RNNs are considered to be black-box models that are hard to interpret.
The lack of understanding of the internal mechanisms makes the models untrustworthy and further limits their extension to other domain applications. 
During model development, the domain experts usually want to better understand the models' forecast.
On the one hand, they tend to learn whether the model's behavior confirms any hypotheses according to existing domain knowledge, which gains confidence in the forecast.
On the other hand, the experts also aim to identify unobserved patterns by exploring different cases to enrich their understanding of the domain problem.
Understanding RNNs also helps model designers better choose models and tune hyper-parameters, adjust the model's architecture, and remove unnecessary features in the input space.

In this work, we are motivated by the requirements of air pollutant forecast. 
The domain researchers have developed RNN models to forecast the concentration of air pollutants.
They aim to understand how RNNs model time-series data and if the model's behaviour matches their domain knowledge, thus increasing confidence in the models. 
As opposed to the symbolic sequences (sentences) used in NLP tasks, the time-series data in our application are sequences of hourly recorded high-dimensional vectors in which each dimension indicates a feature related to the air quality pollutant such as \textit{wind speed} or   
the concentration of $SO_{2}$.
The sequence length ranges between 6 and 72 to record the historical data from the previous 6 to 72 hours.


Existing work on RNN interpretation mainly focuses on NLP tasks. 
For example, LSTMVis~\cite{strobelt2018lstmvis} uses parallel coordinates to visualize the periodic patterns of hidden states in order for experts to verify their hypotheses.
RNNVis~\cite{ming2017understanding} models the input text and hidden units as a bipartite graph and uses co-clustering techniques to cluster the hidden units and text simultaneously to present more structural information learned by the model. 
Both of these two works interpret hidden units by providing their relevant words as language contexts.
However, existing techniques cannot be directly applied in RNNs for high-dimensional time-series forecasts.


First, the high dimensionality of the input sequence makes it difficult to discover relationships between features and hidden states. 
Since different features at various timestamps are correlated, traditional techniques are not applicable as they are not designed to reveal how feature importance changes over time, which hinders the domain experts' ability to analyze if the prediction is consistently generated.
Analyzing this complex temporal multi-dimensional data requires an effective visualization design that can reveal both cross-dimension data relationships and the model's temporal behavior.
Moreover, in real-world applications, the input dimension size may be from hundreds to thousands, and the hidden unit size of the RNN models can also be large.
This requires the visualization methods to have good scalability in order to demonstrate the distribution and complex relationships of hidden units and input features.
% \zw{try to illustrates challenges using our domain words like spatio-temporal, multi-dimensional, multi-scale, comparative.}

% Second, as RNN models capture and utilize the temporal sequence information for prediction, the hidden states are affected by both the features at current time step and the hidden units from previous time steps. 
% This adds an additional complexity in both interpretation and system scalability.


% To address these challenges, We propose a visual analytics system to help domain experts to understand RNN in high-dimensional time series forecast tasks by uncoverng the relationship between input features and hidden states. 
% Specifically, we develop a co-clustering view (\zw{Fig.~\ref{fig:teaser}(xxx)?}) to visualize the relations between hidden units and input features and support flexible interactions to allow users focusing on interested features or feature value ranges.
% We also propose a sequence view (\zw{Fig.~\ref{fig:teaser}(xxx)?}) to summarize and compare the temporal hidden unit and feature importance along the sequences with good scalability.

To address these challenges, we propose MultiRNNExplorer, a visual analytics system to help domain experts to understand RNNs in high-dimensional time-series forecasts from two aspects: the model mechanism and the feature importance. In understanding model mechanism, we estimate the overall response between hidden units and features by measuring the distribution variation of hidden units' output across different feature value selections. Then we generate response embedding for both hidden units and features, and cluster the hidden units and features respectively to reveal the high-level knowledge captured by the models. 
To facilitate case-based exploration, we use gradient-based techniques to measure the feature importance.  The gradients at all timestamps are calculated for all cases to measure the local importance of features. 
Then a visualization interface with coordinated views is designed, enabling the users to interactively explore the model behavior. 
Specifically, we design a Cluster View (Fig.~\ref{fig:teaser}A) to visualize the hidden units' response to features. 
The hidden units are visualized in a distribution chart and the features are visualized as a glyph. Then the Feature Importance View (Fig.~\ref{fig:teaser}C) summarizes the temporal importance of all features. Finally, the individual level behaviors of RNNs are visualized in a \UC{Individual View} (Fig.~\ref{fig:teaser}D) which visualizes the feature trend, cluster importance trend and the top features for each individual case.

Our contributions can be summarized as follows:
\begin{itemize}
  \item A new method for estimating the neuron response to high-dimensional time-series data. 
%   \zw{A main contribution? should mention this earlier.}
  \item A visual analytics system that explores, interprets, and compares RNNs models in high-dimensional time-series forecast tasks. 
%   \zw{more details. any new designs?}
  \item Case studies using real-world datasets to demonstrate the effectiveness of the proposed system and interesting insights revealed from the case studies. For example, in the training process, the models learn the \QM{most direct features in the first and then gradually learn the features which indirectly influence the forecast}.
%   \zw{more details. what interesting insights?}.
\end{itemize}





